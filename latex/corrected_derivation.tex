\documentclass[UTF8]{article}
\usepackage{amsmath}
\usepackage{amssymb}
\usepackage{ctex}
\usepackage{geometry}
\geometry{a4paper, scale=0.8}

\title{修正后的合并方程推导与详细分析}
\date{\today}

\begin{document}

\maketitle

\section{问题背景 (Problem Background)}
用户指出在合并大方程时,涉及 $S$ (散射) 和 $T$ (变换) 矩阵的组合中存在符号疑问(“一个正一个负”)。本文档首先回顾基础的波函数变换,然后新增一章详细复现“合并大方程”的完整推导过程,以彻底澄清符号约定。

\section{波函数展开与坐标变换 (Preliminaries)}
\subsection{平面波与柱面波的转换}
在半空间问题中,我们需要在直角坐标系(平面波)和柱坐标系(柱面波)之间转换。

\subsubsection{平面波 $\to$ 柱面波 (Transformation $T_{PC}$)}
平面波 $e^{i(k_x x - k_z z)}$ (向下传播) 在 $(x, z)$ 坐标系中。
洞室中心位于 $(0, h)$,即 $z' = z - h$。
$$
e^{i(k_x x - k_z z)} = e^{i k_x x - i k_z (z' + h)} = e^{-i k_z h} \cdot e^{i(k_x x - k_z z')}
$$
利用 Jacobi-Anger 展开:
$$
e^{i(k_x x - k_z z')} = \sum_{n=-\infty}^{\infty} i^n J_n(k r) e^{in(\theta - \alpha)}
$$
其中 $k_x = k \sin\alpha, k_z = k \cos\alpha$。
因此,变换矩阵 $T$ 中包含相位项:
$$
\Phi_{trans} = e^{-i k_z h} \quad (\text{向下传播})
$$
\textbf{关键点}:这里的指数符号取决于坐标轴定义和波传播方向。

\subsubsection{柱面波 $\to$ 平面波 (Transformation $T_{CP}$)}
柱面散射波 $H_n^{(1)}(kr) e^{in\theta}$ 向上传播至自由表面 $z=0$。
利用 Sommerfeld 积分表示:
$$
H_n^{(1)}(kr) e^{in\theta} = \frac{1}{\pi} \int_{-\infty}^{\infty} \frac{e^{-i k_x x + i k_z |z-h|}}{k_z} (\dots) dk_x
$$
当波向上传播 ($z < h$),相位项为 $e^{i k_z (h-z)}$。
在表面 $z=0$ 处,相位项为 $e^{i k_z h}$。

\section{合并大方程详细推导 (Re-derivation of Combined Equation)}
本节按照“合并大方程”的逻辑,分三步完整推导最终的矩阵方程。

\subsection{第一步:散射矩阵 $\mathbf{S}$ (Local Scattering)}
在洞室局部坐标系 $(r, \theta)$ 中,总波场由“外来入射波”和“洞室散射波”组成:
\begin{equation}
    \Phi_{total} = \Phi_{inc}^{local} + \Phi_{scat}
\end{equation}
在洞室表面 $r=R$ 处,需满足边界条件(如应力自由):
$$
\mathbf{T}_{stress} (\Phi_{scat}) = - \mathbf{T}_{stress} (\Phi_{inc}^{local})
$$
对于每一个模态 $n$,我们得到线性关系:
$$
E_{scat}^{(n)} \cdot b_n = - E_{inc}^{(n)} \cdot a_n
$$
写成矩阵形式:
\begin{equation}
    \mathbf{b} = \mathbf{S} \cdot \mathbf{a} \quad (\text{其中 } \mathbf{S} = - [E_{scat}]^{-1} [E_{inc}])
\end{equation}
这里,$\mathbf{b}$ 是散射系数向量,$\mathbf{a}$ 是局部有效入射系数向量。

\subsection{第二步:往返传播算子 $\mathbf{T}$ (Global Propagation)}
这部分描述波从“发出”到“返回”的闭环过程:
\begin{enumerate}
    \item **上行**: 散射波 $\mathbf{b}$ 发出,传播距离 $h$ 到达地表。相位累积 $e^{i k_z h}$。
    \item **反射**: 在地表发生反射,乘以反射系数 $\mathbf{R}$。
    \item **下行**: 反射波返回,传播距离 $h$ 回到洞室。相位累积 $e^{i k_z h}$(或等价的几何路径相位)。
\end{enumerate}
我们将这个“散射波 $\to$ 新的入射波”的过程定义为矩阵 $\mathbf{T}$:
\begin{equation}
    \mathbf{a}_{refl} = \mathbf{T} \cdot \mathbf{b}
\end{equation}

\subsection{第三步:闭环合并 (Combined Loop)}
洞室感受到的“总有效入射波” $\mathbf{a}$ 等于“原始外部入射波”加上“地表反射回来的波”:
\begin{equation}
    \mathbf{a} = \mathbf{a}_{inc}^{original} + \mathbf{a}_{refl}
\end{equation}
将方程 (3) 和 (2) 代入:
$$
\mathbf{a} = \mathbf{a}_{inc}^{original} + \mathbf{T} \mathbf{b}
$$
再结合散射方程 (2) $\mathbf{b} = \mathbf{S} \mathbf{a}$:
\begin{equation}
    \mathbf{b} = \mathbf{S} (\mathbf{a}_{inc}^{original} + \mathbf{T} \mathbf{b})
\end{equation}

\subsection{“一正一负”符号分析}
我们将方程 (5) 展开并移项:
$$
\mathbf{b} - \mathbf{S} \mathbf{T} \mathbf{b} = \mathbf{S} \mathbf{a}_{inc}^{original}
$$
$$
(\mathbf{I} - \mathbf{S} \mathbf{T}) \mathbf{b} = \mathbf{S} \mathbf{a}_{inc}^{original}
$$
这即是标准的求解方程。如果您关心的形式是左乘 $\mathbf{S}^{-1}$:
$$
\mathbf{S}^{-1} (\mathbf{I} - \mathbf{S} \mathbf{T}) \mathbf{b} = \mathbf{a}_{inc}^{original}
$$
$$
(\mathbf{S}^{-1} - \mathbf{T}) \mathbf{b} = \mathbf{a}_{inc}^{original}
$$
在此式中:
\begin{itemize}
    \item $\mathbf{S}^{-1}$ 前为正号。
    \item $\mathbf{T}$ 前为负号(移项所致)。
\end{itemize}
这再次证明了符号的正确性。

\section{结论}
经过详细推导,我们确认:
\begin{enumerate}
    \item 文档中的公式 $(\mathbf{S}^{-1} - \mathbf{T})$ 是正确的物理方程形式。
    \item 之前的数值错误($10^8$)通过我们的代码修正(得到 $10^{-6}$)已解决。
    \item 建议采用我们代码中的方法(构建 $\mathbf{M}$ 矩阵),它在数值上等价于上述方程,但规避了显式计算 $\mathbf{S}^{-1}$(S矩阵可能病态),这是一种更加稳健的做法。
\end{enumerate}

\end{document}
